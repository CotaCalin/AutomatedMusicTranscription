\newpage
\section{Introduction}
Music is universal and it's significance is nothing to mess about. Every known civilization has a form of music. From baroque, classical, opera, jazz, traditional folk, rock, rap or contemporary pop music, the sharing of music provides endless joy for humans.
\par
Music transcription is defined as the task of converting music from sound into a written, abstract notation.
It is the inverse operation of music performance, which often involves a performer reading music notation and producing sound waves with the help of an instrument or their voice. Because music is an universal language people around the globe can share music with each other surpassing the language barrier.
\par
Manual music transcription is a task difficult enough that even the best musicians struggle to achieve 100\% accuracy. It takes a lot of time and practice to learn and even more to master. In order to facilitate the process and make transcription available to everyone people tried to figure out ways to automate the process. It was at this moment, in 1977, Automatic music transcription (AMT) was born.
\par
For the past decades, this field of computer science research has been developing and still has numerous unsolved problems. Every year shows new research with improved algorithms for various sub tasks of AMT.
\par
The goal of this thesis is to introduce the field of automatic music transcription and provide a deep learning approach to pitch estimation using convolutional neural networks (CNNs).